\section{Security Analysis}
\label{sec:security}

For detailed security analysis of our scheme, let suppose that Alice is the
owner of a small enterprise named Digitals. Alice wants to outsource confidential data files $\mathcal{F} = \langle \mathcal{F}_1,
\mathcal{F}_2, \ldots, \mathcal{F}_n \rangle$ to a cloud server, owned by Eve - an untrusted third party providing public cloud services. 
Alice also wants Bob, HR manager at Digitals, to search the outsourced data using arbitrary search queries. 
Alice ensures privacy and confidentiality of $\mathcal{F}$ by
first encrypting $\mathcal{F}$ and corresponding index files $\mathcal{I}
= \langle \mathcal{I}_1, \mathcal{I}_2, \ldots, \mathcal{I}_n \rangle$ with
$\sigma_{pk}$ and then outsourcing them to Eve.

\subsection{Man-in-the-middle-attack}

Suppose a malicious user $M$ intercepts a user query $Q$. Since $Q$ is encrypted with $\sigma_{sk}$, $M$ cannot learn any useful information by simply intercepting the search request. 
However, $M$ can manipulate $Q$ to produce $\acute{Q}$ by replacing $k\omega$ in $Q$ with his own $\acute{k}\omega$ encrypted with $\sigma_{pk}$. 
Eve cannot identify if the query is been altered during transmission. Eve will evaluate $\acute{Q}$ and sends $\mathcal{R}_q$ to the user.
Similar to Eve, user will not notice that $\mathcal{R}_q$ is the outcome of an altered query. %User may notice unwanted information in the result but won't notice the missing records due to replacement of his keyword by $M$.
However, homomorphic encryption will restrain $M$ from compromising privacy of the query $Q$. For a successful man-in-the-middle-attack, $M$ would need $\sigma_{sk}$ which is only accessible to Alice. Thus, confidentiality and privacy of $Q$, $F$ and $I$ remain intact in this and subsequent attack scenarios. 

\subsection{Malicious cloud service provider}
Suppose Eve has malicious intentions and wants to know about $\mathcal{I}$ and
user queries. As Eve does not have $\sigma_{sk}$, she cannot decrypt $\mathcal{I}$. 
Also, all comparisons between the bloom filter of the user
query and $\mathcal{I}$ are being done using the homomorphic properties of the Pascal 
Paillier~\cite{pascal} algorithm, so Eve can neither determine the outcome of the
comparisons, nor can she determine any search pattern for subsequent queries from different users.
This means that $\mathcal{F}$ and $\mathcal{I}$ are also secured against any insider attacks. 
A malicious employee would need $\mathcal{K}_S$ and $\sigma_{sk}$ to decipher $\mathcal{F}$ and $\mathcal{I}$.  

\subsection{Malicious user and cloud service provider}
Suppose a malicious user $M$ having some background information of Digitals teams up with Eve to 
try and compromise confidentiality and privacy of $Q$, $F$ and $I$. To successfully infer any information from the outsourced data or search query they would need $\mathcal{K}_S$ and $\sigma_{sk}$. Since, these key are with Alice, both $E$ and Eve are restrained in their collective effort and cannot learn any useful information from the encrypted outsourced data. 
%They
%both cannot learn anything about the user query and the indexes. Malicious user cannot decrypt the user query and Eve cannot know the comparisons output.

%\subsection{Malicious user�s random query}
%
%Let malicious user $M$ having $\sigma_{pk}$, creates a semantically correct query
%and sends to Eve for evaluation. Eve will perform comparisons and will return
%$\mathcal{R}_q$ but $M$ won�t be able to determine $\mathcal{M}_q$ as he does not
%have the $\sigma_{sk}$ to decrypt $\mathcal{R}_q$ returned by Eve. He will
%receive $\mathcal{R}_q$ but he won�t be able to learn any information out of it.
