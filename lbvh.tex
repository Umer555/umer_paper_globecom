\section{Overview}
\label{sec:lbvh}

Using the open-source 
Sirikata system~\cite{ewencp2012} as a starting point, this paper
optimizes the download and rendering cost of aggregates by exploiting
content coherence. Two key
algorithmic improvements enable this optimization:

%\begin{figure}
%\centering
%\includegraphics[width=0.5\textwidth]{fig/gat.pdf}
%\caption{Sirikata's BVH}
%\label{fig:gat}
%\vspace{-8pt}
%\end{figure}


(A) \textbf{Grouping objects into aggregates based on visual similarity.}  
Existing approaches group
together objects in a BVH to optimize query performance and ignore the
resulting mesh complexity or size of the BVH's internal nodes. 
Many scenes have collections of very similar objects: trees grouped into a 
forest, bricks forming a wall, and similar houses
grouped into a suburb. As Figure \ref{fig:bvh_optimization} shows,
considering the visual similarity of objects can lead to aggregates
where highly similar objects can be replaced (or \emph{deduplicated}) 
with one representative object
instanced multiple times.  The resulting aggregate meshes can
significantly reduce file size and rendering complexity.  Sections
\ref{sec:bvhconstruction} and \ref{sec:generation} describe algorithms
for constructing similarity-aware BVHs and similar mesh deduplication.

(B) \textbf{Simplifying instanced meshes efficiently into small output
  mesh files.} User-generated virtual worlds require an open 3D
format, such as COLLADA. Most such formats describe the 3D model as an
instanced mesh, where a model consists of a set of submeshes which are
instantiated one or more times to create the overall mesh.  This
reduces the download and memory cost for clients compared to an
expanded version of the same mesh. It also encourages modification and
reuse of meshes by users because they are easier to edit: changes to a
submesh, such as a window on a house, are made across all instances.
Aggregates can also increase instancing, because aggregates group
similar objects together and represent them as instances of the same
object.

However, because quadric simplification requires all instances in the
mesh to be expanded out into a single instance, it can drastically
increase the file size of the resulting mesh. Results in
Section~\ref{sec:eval}, for example, show that on instanced meshes,
using existing simplification algorithms to reduce the number of triangles by 80\% 
can double the file size. Section \ref{sec:simplification} presents an
instance-aware simplification algorithm that strictly reduces file
size as it simplifies a mesh. 

%(C) \textbf{Reconfiguring aggregate meshes to better manage GPU
%  texture resources.}  Combining many meshes into aggregates can
%result in each aggregate having large numbers of seperate textures,
%which becomes a GPU bottleneck since each texture requires a separate
%draw call.  Section \ref{sec:textures} describes an approach towards
%texture management that ensures each aggregate mesh has only a single draw call
%for all of its textures.

