\section{IAS: Instance-Aware Simplification}
\label{sec:simplification}

When an instanced mesh is expanded
into an indexed triangle mesh, it results in duplicated geometry
and a much larger model. Simplifying such an expanded mesh is a
one-way transformation: once edges in one copy of a submesh are
collapsed, that copy cannot be easily factored back into a single
submesh. As a result, a simplified instanced mesh can reduce graphical
complexity while {\em increasing} file size. Our instance-aware
simplification algorithm (IAS) addresses this problem.

Before we explain IAS, we
provide some background on quadric simplification. 
We use quadric simplification as a base algorithm due to
its excellent speed/performance tradeoff and support for multiple
moderately complex objects~\cite{luebke01developer}, both of which are
important for games and virtual worlds. 

%@article{luebke01developer,
% author = {Luebke, David P.},
% title = {A Developer's Survey of Polygonal Simplification Algorithms},
% journal = {IEEE Comput. Graph. Appl.},
% issue_date = {May 2001},
% volume = {21},
% number = {3},
% month = may,
% year = {2001},
% issn = {0272-1716},
% pages = {24--35},
% numpages = {12},
% url = {http://dx.doi.org/10.1109/38.920624},
% doi = {10.1109/38.920624},
% acmid = {618821},
% publisher = {IEEE Computer Society Press},
% address = {Los Alamitos, CA, USA},
%} 


\subsection{Background: Quadric Simplification}

Quadric mesh simplification executes in two
phases. During the initialization phase, the algorithm assigns an error
quadric, $Q$, to each vertex, $v$. $Q$ is computed
on the basis of the planes (triangles) neighboring $v$, and is given by
$Q=\sum_{p \in planes(v)} Q_p$, where
$Q_p$ is the quadric for plane p and is computed as:
\begin{equation}
\label{eq:Qp_equation}
 \mathbf{Q_p} = area(p) . \left[ \begin{array}{cccc}
a^{2} & ab & ac & ad \\
ab & b^{2} & bc & bd \\
ac & bc & c^2 & cd \\
ad & bd & cd & d^2                    \end{array} \right] 
\end{equation}
Here $a, b, c$ and $d$ are the normalized coefficients of the equation
$ax+by+cz+d=0$, which defines the plane p, while $area(p)$ is the area
of the triangle corresponding to plane p.
With this formulation, given a vertex $w$, $w^TQw$
is a measure of the distance of vertex $w$ from the set of planes
in planes($v$).

For every edge $(v_1,v_2)$, assuming that quadrics
$Q_1$ and $Q_2$ are associated with
$v_1$ and $v_2$, it then computes an optimal
contraction target $\bar{v}$ for which the cost is given by
\begin{equation}
\label{eq:vbar_cost_equation}
cost(\bar{v}) = \bar{v}^T (Q_1+Q_2) \bar{v}. 
\end{equation}
Assuming $K = (Q_1+Q_2)$, $\bar{v}$ is
computed as:
\begin{equation}
\label{eq:vbar_equation}
 \mathbf{\bar{v}} = \left[ \begin{array}{cccc}
k_{11} & k_{12} & k_{13} & k_{14} \\
k_{12} & k_{22} & k_{23} & k_{24} \\
k_{13} & k_{23} & k_{33} & k_{34} \\
0 & 0 & 0 & 1                    \end{array} \right]^{-1}
\left[ \begin{array}{c}
0 \\
0 \\
0 \\
1                    \end{array} \right] 
\end{equation}
In the edge collapse phase, edges are collapsed iteratively in
increasing order of these error values, with the cost of vertices
neighboring $v_1$ and $v_2$ updated after each
collapse.

\subsection{Instance-aware Simplification Algorithm}
\label{sec:ias-algorithm}

Instance aware simplification converts the hierarchical representation
of an instanced mesh $M$ into a flat representation of a list of
instances. The transform associated with each instance is the product
of all transforms from the root of the hierarchy to the instance
itself. Each instance indexes into a list of a submeshes, which
contain geometry information in indexed triangle mesh format.
Formally, $M=\{I_1, I_2, I_3, ..., I_n\}$, where $n$ is the number of
instances in the mesh. Each instance $I_j=<T_j,S_k>$, where $T_j$ is
the transformation matrix associated with $I_j$ and $S_k$ is the
submesh $I_j$ references.  The submesh $S_k$ is from a list of
submeshes $\{S_1, S_2, ..., S_m\}$, such that $1 \le k \le m$ and $m
\le n$.  Each submesh $S_k$ consists of a set of vertices and a set of
triangles referencing those vertices.

%A naive approach to simplify an instanced mesh would be to run
%quadric simplification on each submesh and maintain
%the intermediate LODs for each submesh.  Then to simplify
%the overall mesh to a target triangle count or file size, choose an appropriate
%LOD for each submesh. However, this leaves open the
%question of how to choose these LODs.
% which can be
%complicated if the submesh is instanced using wildly
%varying transformations.

%Instance-aware simplification applies quadric
%simplification to the underlying submeshes but accounts for the fact
%that each submesh may be instantiated multiple times and transformed
%in different ways to create the overall mesh.

%One way to do this intuitively is to iterate through the list of mesh
%instances, compute the cost of each edge in the overall mesh, and add
%it back to the total cost of the underlying submesh edge. Then, order
%each submesh edge by its total cost and collapse edges in increasing
%order of their total cost. However, this approach does not help us
%find an optimal contraction target for a given submesh edge.

IAS uses the observation that the error quadric $Q$
associated with a vertex $v$ is derived from the set of planes
neighboring $v$. In an instanced mesh, therefore, $Q$
for a submesh vertex $v$ can be computed by accounting for all
the neighboring planes that exist in all instances of the
submesh. Suppose $T$ is the transformation matrix for a given
instance of a submesh, and $v$ is a submesh vertex which maps
to $x$ in that instance. Since $x=Tv$, we can write the
distance of $x$ from its set of neighboring planes as
\begin{equation}
x^TQx=(Tv)^TQ(Tv)=v^TT^TQTv
\end{equation}
where $Q$ is computed from the neighboring planes in that instance.

Then, $(T^TQT)$ is the error quadric giving the distance of
the submesh vertices from the neighboring planes in the
instance. Summing it over all instances, the error for a submesh
vertex is 
\begin{equation}
E_s = \sum_{i \in I}{T_i^TQ_iT_i}
\end{equation}
 where $I$ is the set of of instances of the submesh,
$T_i$ is the transform associated with instance $i$, and
$Q_i$ is the quadric computed for the instantiated vertex.

Using this new quadric, we can find
optimal contraction targets for submesh edges and simplify an instanced mesh
down to a target triangle count as follows:

\begin{enumerate}
\item For each instance $i$ applying $T_i$ to submesh $S_i$:
      \begin{enumerate}
        \item For each triangle $t$ in submesh $S_i$:
        \begin{enumerate}
        \item Transform $t$ by applying $T_i$ to each of its vertices.
        \item Compute $Q_p$, the error quadric for the transformed 
               triangle, using Equation \ref{eq:Qp_equation}.
        \item Compute the error quadric for the untransformed triangle 
               $t$ as $T_i^TQ_pT_i$
               and add it to the error quadrics for each of 
               $t$'s untransformed vertices.
        \end{enumerate}
      \end{enumerate}

    \item In each submesh $S_i$, compute the optimal
      contraction target $\bar{v}$ and its cost for each
      submesh edge $(v_1,v_2)$ using Equations
      \ref{eq:vbar_equation} and \ref{eq:vbar_cost_equation}, where
      $Q_1$ and $Q_2$ are the submesh error quadrics
      for $v_1$ and $v_2$ respectively.

    \item Collapse submesh edges in increasing order of their cost.
      Compute how many triangles become degenerate after each collapse
      and decrement the number of triangles in the model by that times
      the number of instances of the submesh.  At each step, since
      only a submesh edge is collapsed, the cost has to be updated
      only for neighboring vertices in that submesh.

\end{enumerate}

As the algorithm proceeds, all edges of a submesh may collapse due to which the
submesh may not remain visible. This is particularly a problem 
when there are many instances of a very simple submesh, such as the leaves
of a tree: a single edge collapse in the submesh may cause all
leaves to disappear. To handle this case, IAS employs a variation of 
stochastic simplification \cite{cook2007stochastic}. First, it does not collapse
an edge if that causes a submesh to disappear. Second, to deal with many instances
of a simplified submesh, it eliminates some instances of the submesh, 
while scaling up the
remaining instances to preserve surface area. To do this, it computes the number of
instances of the submesh needed to achieve the target triangle count. 
It keeps and scales up that number of instances while removing the rest.
In doing so, it prioritizes keeping those instances 
which, if scaled up, remain within the original 
bounding box of the model. 


Finally, IAS handles boundary edges in a special manner due to their
importance. A boundary edge is an edge that exists in only one
triangle.  IAS considers an edge to be a boundary edge as long as it
is a boundary edge within its submesh. For each boundary edge, 
IAS generates a perpendicular constraint
plane running through the edge. It then computes the quadric for this
constraint plane, weights it by the length of the edge and adds it to
the quadrics for the endpoints of the edge. This results in much
better results than simply marking such edges as incollapsible since
it still allows small boundary edges to be collapsed in place of
other longer edges.

\setlength{\tabcolsep}{.36667em}
\begin{table*}
\centering
{\small
\begin{tabular}{l|rrrr|rrr|rrr}
\toprule[1.5pt]  Model    &
\specialcell{Submeshes} & Instances & \specialcell{Instances per\\ submesh (IPS)} & Triangles  &
\specialcell{Time\\(Quadric)} & \specialcell{Time\\(IAS)} & \specialcell{Time\\ Reduction} &
\specialcell{Size\\(Quadric)} & \specialcell{Size\\(IAS)} & \specialcell{Size\\ Reduction}  \\ \midrule[1pt]
Bunny           &    1 &     1 &      1 &    20,000 & 390 ms & 1890 ms & 0.2:1 & 235 KB & 235 KB & 1:1  \\
Patio Chair     &    6 &    68 &  11.33 &     2,240 & 40 ms  & 43 ms   & 0.9:1   & 32 KB  & 31 KB  & 1:1    \\
% Stonehenge   &   59 &    80 &   1.36 &    10,061 & 140 ms & 608 ms  & -334\% & 379 KB & 373 KB & 1.6\%    \\
Maple Tree      &   18 &  9324 &    518 & 1,818,074 & 13 s   &  5 s    & 2.6:1   & 17 MB  & 110 KB & 154:1 \\
Village         &   79 & 13523 & 171.18 & 1,254,696 & 30 s   & 17 s    & 1.8:1   & 31 MB  & 6 MB   & 5:1  \\ \bottomrule[1.5pt]
\end{tabular}
}
\caption{Performance comparison of IAS and Quadric simplification for
  the three models shown in
  Figures~\ref{fig:densemaple}--~\ref{fig:patiochair} as well as the Stanford
  bunny. Each model was simplified to 20\% of its original triangle count.
  Instance-aware simplification outperforms quadric simplification in
  speed for highly instanced meshes while simultaneously better reducing file
  sizes.}
\label{tab:ias_results}
\end{table*}



\section{Results}
\label{sec:simpl_eval}


\begin{figure*}
\centering
\subfigure[Quadric (20\% triangles, 17 MB)] {
\includegraphics[width=1.75in]{fig/models/densemaple_exp00.png}
\label{fig:densemaple-expanded}
}
\hspace{.3in}
\subfigure[Original, 546 KB]{
\includegraphics[width=1.75in]{fig/models/densemaple00.png}
\label{fig:densemaple-original}
}
\hspace{.3in}
\subfigure[IAS (20\% triangles, 110 KB)]{
\includegraphics[width=1.75in]{fig/models/densemaple_instanced00.png}
\label{fig:densemaple-ias}
}
\caption{Maple tree (518 instances per submesh) from
  Table~\ref{tab:ias_results} simplified to 20\% of the original
  triangle count. Each leaf is an instance of the same submesh. Using
  a simplification algorithm that is not instance aware causes the
  file to increase in size by a factor of 30. Note how IAS preserves
  the appearance of individual leaves while reducing file size by
  80\%.}

\label{fig:densemaple}
\end{figure*}



\begin{figure*}
\centering
\subfigure[Quadric (6MB)] {
\includegraphics[width=1.75in]{fig/stills/village_quadric_6MB.png}
\label{fig:screen-cuts}
}
\hspace{.3in}
\subfigure[Original (16 MB)] {
\includegraphics[width=1.75in]{fig/stills/village.png}
\label{fig:original_village}
}
\hspace{.3in}
\subfigure[IAS (6 MB)]{
\includegraphics[width=1.75in]{fig/stills/village_ias_0.png}
\label{fig:screen-aggs}
}
\caption{Village (171 instances per submesh) from
  Table~\ref{tab:ias_results} simplified to 6MB in size. Quadric
  simplification removes all of the roads, reduces all of the homes to
  their roofs, and turns trees into simple crossed triangles. Although
  each road segment is small, their repeated, regular presence makes
  them a visually important feature of the scene.}
\label{fig:village}
\end{figure*}

%%%%%%%%
\begin{figure*}
\centering
\subfigure[Quadric, 32 KB] {
\includegraphics[width=1.5in]{fig/models/patiochair3_exp00.png}
\label{fig:patiochair-quadric}
}
\hspace{.5in}
\subfigure[Original, 92 KB]{
\includegraphics[width=1.5in]{fig/models/patiochair300.png}
\label{fig:patiochair-original}
}
\hspace{.5in}
\subfigure[IAS, 31 KB]{
\includegraphics[width=1.5in]{fig/models/patiochair3_instanced00.png}
\label{fig:patiochair-ias}
}
\caption{Patio chair (11.3 instances per submesh) from
  Table~\ref{tab:ias_results} simplified to approximately one third of
  the file size. IAS preserves the slats while quadric simplification
  removes some of them.}
\label{fig:patiochair}
\end{figure*}
%%%%%%%



%\begin{figure*}
%\centering
%\subfigure[Quadric (20\% triangles, 379 KB)] {
%\includegraphics[width=1.75in]{fig/models/Stonehenge_exp00.png}
%\label{fig:stonehenge-quadric}
%}
%\hspace{.3in}
%\subfigure[Original, 1.1 MB]{
%\includegraphics[width=1.75in]{fig/models/Stonehenge00.png}
%\label{fig:stonehenge-original}
%}
%\hspace{.3in}
%\subfigure[IAS (20\% triangles, 373 KB)]{
%\includegraphics[width=1.75in]{fig/models/Stonehenge_instanced00.png}
%\label{fig:stonehenge-ias}
%}
%\caption{Stonehenge mesh from Table~\ref{tab:ias_results} simplified to
%  20\% of the original triangle count. The visual quality of
%  IAS is similar to quadric simplification.}
%\label{fig:stonehenge}
%\end{figure*}

%%%%%%%%%%%%%%%%%%%%%%%%%%%%%

%\begin{figure*}
%\centering
%\subfigure[Quadric (20\% triangles, 235 kB)] {
%\includegraphics[width=1.75in]{fig/models/bunny_4000_00.png}
%\label{fig:bunny-quadric}
%}
%\hspace{.3in}
%\subfigure[Original (??? kB)]{
%\includegraphics[width=1.75in]{fig/models/bunny_00.png}
%\label{fig:bunny-original}
%}
%\hspace{.3in}
%\subfigure[IAS (20\% triangles, 235kB)]{
%\includegraphics[width=1.75in]{fig/models/bunny_q4000_00.png}
%\label{fig:bunny-ias}
%}
%\caption{The Stanford bunny (IPS=1) simplified to 20\% of the original
%  triangle count. The visual quality using both IAS and quadric
%  simplification is similar.}
%\label{fig:bunny}
%\end{figure*}

Table~\ref{tab:ias_results} shows results for
instance-aware simplification (IAS) on four sample models with
different degrees of instancing, measured by their instance count
per submesh (IPS). Figures~\ref{fig:densemaple}--\ref{fig:patiochair} show the 
visual results.  We do not show visual results for the Stanford bunny because
IAS devolves to standard quadric simplification, so it is visually
identical.

The dense maple tree and village (Figures~\ref{fig:densemaple}
and \ref{fig:village}) are
highly instanced meshes with over a million triangles each and on
average over 100 instances per submesh.  Simplification with IAS
results in a much smaller output mesh that has greater visual quality.
In the case of the maple tree, IAS reduces file size by a factor of 5
and maintains the visual quality of the leaves; quadric simplification
distorts the leaves while increasing file size by a factor of 30. In
the case of the village, IAS achieves significantly better visual quality
for the same target triangle count.


\begin{figure}
\centering
\includegraphics[width=.45\textwidth]{fig/plot-data/tree-triangle-filesize.pdf}
\caption{For the tree model in Figure~\ref{fig:densemaple}, IAS is able
to maintain the same triangle count as quadric simplification at one hundredth
the file size. Even after 99\% simplification, the quadric simplified model
is larger than the original file.}
\label{fig:size-plots}
\vspace{-8pt}

\end{figure}

Figure~\ref{fig:size-plots} shows the size of the tree model under IAS
and quadric simplification. Simply applying quadric simplification
expands the mesh to 36MB. Reducing it to 1\% of its original triangles
is 891KB, larger than the original file size. In contrast, IAS
strictly decreases file size as LOD reduces.

Using IAS, simplification of these heavily instanced models proceeds
faster than quadric simplification, because IAS's edge
collapse operates only on submesh edges.
There are fewer submesh edges than edges in the overall mesh
and collapsing a single submesh edge effectively collapses multiple
edges in the overall mesh.  

Submeshes in the patio chair are not highly instanced
(IPS=11.33). IAS takes slightly longer than quadric
simplification since it must perform additional matrix multiplications
to compute quadrics for the transformed submeshes. IAS
recognizes the visual importance of the slats in the chair back (an
instanced submesh) and so maintains them, while quadric simplification
produces holes.

Finally, the Stanford bunny mesh has exactly one instance per submesh. 
It takes longer to simplify it using IAS, but the visual
error remains the same. For such meshes with few instances per
submesh, existing algorithms are more
efficient.  

\subsection{Running IAS on a model dataset}
\label{sec:ias_on_database}

In order to test if IAS is generalizable to all user-generated 3D models, we
tested IAS on a dataset of 748 models uploaded by a group of students while
building their own virtual worlds. These models
vary widely in complexity from 1 to 1.8 million triangles, and from 
1 to almost 1000 IPS (Figure~\ref{fig:ips_meshes}). We use this dataset 
because it contains a representative sample of models used to construct virtual
worlds.

Each model in the dataset was simplified to 20\% of its original triangle
count using both quadric simplification and IAS.
The solid blue line in Figure \ref{fig:ias_filesize_reduction} compares the file size
of the resulting simplified models using the two algorithms. 
Unexpectedly, in almost 20\% of models, quadric simplification actually
results in smaller file sizes than IAS. Most models where quadric 
simplification outperforms IAS have a low degree of instancing,
measured by its IPS. On the other hand, if the IPS is higher than 1.75, 
IAS performs worse in only a handful of cases. 
Therefore, IAS switches to quadric simplification
whenever the IPS of the input mesh is less than 1.75. 

The dashed red line in Figure \ref{fig:ias_filesize_reduction} shows that, with this optimization, 
quadric simplification outperforms IAS in only 1\% of models. 
In these models, the submeshes are so small that it is more
efficient to store them in an expanded format instead of storing their instance
transformations. Among the remaining models, about 16\% see more than
20\% reduction in file size using IAS. Some complex models even see more than 99\%
reduction in file size using IAS compared to quadric simplification. 
On the average, IAS results in 57\% smaller simplified file size than quadric 
simplification.

Finally, the dashed black line in Figure \ref{fig:ias_filesize_reduction} shows that on almost 98\%
of models, IAS also achieves the same or lower geometric error compared to quadric simplification. 
In the remaining 2\% of models, while the percentage error introduced by IAS relative to quadric 
simplification is sometimes large, the absolute error value remains small.

\begin{figure}
\centering

\includegraphics[width=0.40\textwidth]{fig/instance_count_cdf_csv.pdf}
\caption{CDF showing the distribution of instances per submesh in our dataset.}
\label{fig:ips_meshes}
\vspace{-8pt}
\end{figure}

\begin{figure}
\includegraphics[width=0.45\textwidth]{fig/plot-data/before_inst_count_opt_csv_after_inst_count_csv.pdf}

\caption{CDFs showing percentage reduction in file size and geometric error achieved by IAS compared to
quadric simplification. Metro's Hausdorff distance is used to measure geometric error.
The dashed red and black lines show the results
if IAS switches to quadric simplification when the instances-to-submeshes ratio is
less than 1.75.} 
\label{fig:ias_filesize_reduction}
\vspace{-10pt}

\end{figure}


\section{Discussion}
\label{sec:simplification_discussion}

If every instance of a submesh has the same scaling, then existing
simplification approaches can be trivially applied by multiplying the
edge collapse cost from one instance by the number of instances. This
approach creates artifacts when there are multiple submeshes and their
instances have very different scalings,
as a large instance of a submesh has greater visual impact
than a small one. In our user uploaded dataset of 748 models, 15\% of
models have this property.

IAS does not compute quadrics across submeshes. It knows less about
the overall mesh, which could result in lower quality outputs.
However, our evaluations in Section~\ref{sec:simpl_eval} show that, in
practice, IAS typically results in higher quality results. This is
mainly because IAS does not allow instances of a submesh to
diverge.  For example, unlike existing algorithms, IAS keeps the
leaves of the tree (Figure~\ref{fig:densemaple}) and the slats of the
patio chair (Figure~\ref{fig:patiochair}) unchanged after
simplification.  Our current IAS implementation does not optimize
other vertex attributes such as texture coordinates.
Extending the quadric to include these attributes is part of
future work.
