\section{Related Work}
\label{sec:related}

Traditionally, confidential information is protected using access control
mechanisms. This mechanism only works if confidential information is present on
a fully secure, trusted server. But this assumption fails when confidential 
information is outsourced to remote servers in the cloud.

Although cloud service providers (CSP) allow data to be encrypted in order to 
secure it, encryption leads to a couple of problems. First, searching for 
keywords within data encrypted using popular encryption algorithms is not
possible. Second, if you provide the CSP the decryption keys in order to make
search possible, you cannot prevent the CSP from accessing the confidential 
data.

Recent work has focused on solving these problems using a variety of techniques.
These techniques can be broadly classified into two categories.

\textbf{Using Trapdoor Encryption.}
Song et al. \cite{song} proposed a symmetric key based
searchable scheme for encrypted data. Each keyword in the document was encrypted independently
using trapdoors. Their approach proposed ways to search encrypted data using both an
encrypted index as well as searching the original data itself. 
Goh\cite{goh2003secure} proposed encrypted search using bloom filters. They generate trapdoors
against all the keywords in a file and add them to a bloom filter. In this way, for
each file, a single bloom filter is created using trapdoors and stored in
the cloud. To search, a user computes the trapdoor for a keyword and sends it to the cloud
server. The cloud server checks the trapdoor in the bloom filter, and if
it exists, returns the corresponding file identifier. 
Waters et al. \cite{waters2004building} extended the work of Song et al. and proposed a similar
technique to secure audit logs. Audit logs contain sensitive information about a series of
events, actions and actors who are responsible for triggering particular event
or performing an action. Therefore encryption is required for its confidentiality.
When it needs to be searched, a trusted third party issues a trapdoor
for a specific keyword search. All of the above schemes support only
exact keyword matches and rely on trapdoors.

Boneh et al \cite{boneh}
presented the first public key based searchable scheme which enables authorized users having 
the private key to search in the index. This approach still
used trapdoors based indexing and supported only exact keyword matching.

Yu et al. \cite{li} proposed a scheme on encrypted Personal Health Records
(PHR) by using Hierarchical Predicate Encryption (HPE). They also used a 
trusted third party for the distribution of trapdoors. 
Authorized users obtained trapdoors from the trusted third party and then 
submitted them to the CSP for evaluation. This scheme is limited because only 
predefined trapdoors can be used and users cannot model their own queries.

Pal et al. \cite{saibal} proposed an encrypted search scheme using bloom filters. They
also used soundex coding\cite{odell1918soundex} for each word to search words
which are pronounced similarly. This work is complementary to our work. However, it
does not make an effort to reduce the communication cost of returning results to the
client. Kuzu et al.\cite{mehmat} introduced a similarity-based encrypted search scheme,
which used locality sensitive hashing for the similarity score calculation.
This scheme also used trapdoors, hence leaking private matching information
upon which statistical attacks are possible. 


\textbf{Using Homomorphic Encryption.}
Homomorphic encryption is an encryption technique that
allows computations to be carried out on ciphertext, thus generating an 
encrypted result, which when decrypted, matches the result of operations performed 
on the plaintext. An example of homomorphic encryption is the Pascal Paillier 
cryptosystem~\cite{pascal}, which provides two useful properties: 
(i) the product of two ciphertexts will decrypt to the sum of their corresponding plaintexts
and (ii) a ciphertext raised to a constant $k$ will decrypt to the product of the 
corresponding plaintext and the constant.

Pervez et al.~\cite{zeehan} proposed an encrypted data search scheme using an inverted index.
They used homomorphic encryption~\cite{craig} to provide end-to-end privacy. Only
authorized users could execute queries using their personal proxy re-encryption keys in
order to manipulate the index, so a lot of computation is done due to index
transformations. They used a trusted third party to rank search results and model queries.
While this approach avoids using trapdoors, it still supports only
exact keyword matching and has high communication cost.

Finally, CryptDB~\cite{popa2011cryptdb} and MONOMI~\cite{tu2013processing} improve the performance of searching over encrypted data using
a split client/server querying paradigm. By employing several forms of encryption including
symmetric, public-key and homomorphic encryption, they improve querying performance by orders
of magnitude. However, in doing so, they are vulnerable to several kinds of information leakage 
but with low probability.
